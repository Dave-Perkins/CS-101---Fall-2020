\documentclass[12pt]{article}

\title{CS 101 - Computer Science for All - Syllabus}
\author{}
\date{Fall 2020}

\usepackage{natbib}
\usepackage{graphicx}
\usepackage{hyperref}
\usepackage{longtable}

\begin{document}

\maketitle

\paragraph*{Course Website.} \url{https://piazza.com/hamilton/fall2020/cs101}

\paragraph*{Professors} ~\\

\noindent
Tom Helmuth \\
Lecture section 01 (MWF 9:40 -- 10:30)\\
Lab sections L-01 (M 2 -- 4) and L-02 (T noon -- 2) \\
Office: Taylor Science Center 2015B \\
Email: \texttt{thelmuth@hamilton.edu}\\
Office Hours: T 10 -- 11:30am, W 1 -- 2:30pm, F 1 -- 2:30pm \\

\noindent
Mark Bailey \\
Lecture section 02 (TR 10:10 -- 11:25)\\
Lab sections L-03 (T 2 -- 4) and L-05 (R 2 -- 4) \\
Office: Taylor Science Center 3004 \\
Email: \texttt{mbailey@hamilton.edu}\\
Office Hours: ??? \\

\noindent
David Perkins \\
Lab sections L-04 (W 2 -- 4) and L-06 (F 2 -- 4) \\
Office: Taylor Science Center 2016 \\
Email: \texttt{dperkins@hamilton.edu}\\
Office Hours: ???

\paragraph*{Course Description.} The first course in computer science is an introduction to algorithmic problem-solving using the Python programming language. Topics include primitive data types, mathematical operations, structured programming with conditional and iterative idioms, functional abstraction, and objects. Students apply these skills in writing programs to solve problems in domains across the liberal arts. No previous programming experience necessary.

\paragraph*{Teaching Assistant Hours.} Help is available in Science Center 3040 AND ON ZOOM SOMEWHERE. Times:

\begin{quote}
\begin{tabular}{p{0.9in}p{0.6in}l}
Monday & & 7--10pm \\
Tuesday & & 7--10pm \\
Wednesday & 4--6pm & 7--10pm \\
Thursday & 4--6pm & 7--10pm \\
Friday & 4--6pm & \\
Saturday & 4--6pm & \\
Sunday & 4--6pm & 7--10pm \\
\end{tabular}
\end{quote}

TAs are available to answer your questions and help you with your programs. You should not expect to leave TA hours with all of your problems solved. They are available to you as a source of advice and hints, but their duties do not include fixing everything.

\paragraph*{Textbook.} ???

\paragraph{Assessments.} The following assessments will be used to determine your grade, with the given weights.

\begin{quote}
\begin{tabular}{p{1.75in}p{.5in}}
Labs & 10\% \\
Projects & 50\% \\
Weekly Quizzes & 25\% \\
Final Thing??? & 15\% \\
\end{tabular}
\end{quote}

\noindent
\underline{Labs:} Labs make up an important part of this course. We will cover material in labs that is not covered in the rest of class and vice versa. Each lab will be graded by a short online quiz at the end of it covering the material of the lab.

\vspace{0.5cm}

\noindent
\underline{Projects:} You will be given 2 weeks to complete each of the 6 projects in this class.

\vspace{0.5cm}

\noindent
\underline{Weekly Quizzes:} Weekly quizzes will be taken online on your own time on Wednesday or Thursday each week.

\vspace{0.5cm}

\noindent
\underline{Final Thing:} What is this???


\paragraph*{Attendance policy.}You are expected to attend every class, and to arrive before class begins. You may be excused only for college-sanctioned activities.  You must let the professor know about such absences as soon as you are notified.
If you are sick, please email the professor before the class, don't come to class, and take care of yourself.

\paragraph*{Late work policy.} No late work will be accepted except for unusual circumstances.

\paragraph*{Warm-ups.} \textit{Are we doing warm-ups? Are we giving grade/extra credit for them? I (Tom) votes no grade or extra credit.}

\paragraph*{Collaboration and citation policies.} \textit{What do we want here?}

\paragraph*{Computers.} We expect that each student will work on their own computer for this class. If this is a problem, please contact the professor.

\paragraph*{Accommodations.} Any student with a documented disability requesting academic adjustments or accommodations must speak with the professor during the first two weeks of class and provide written documentation of the suggested accommodation from the Dean of Students Office (Allen Harrison, Elihu Root House; ext. 4021). All discussions will remain confidential.


\paragraph*{Calendar.} This will change as needed.

\begin{longtable}{| l | l | p{2.5in} | l |}
  \hline
  \textbf{Week} & \textbf{Week Date} & \textbf{Topics} & \textbf{Notes}\\\hline\hline
  0 & Jan. 22 & intro  & \\\hline  
  1 & Jan. 27 & types, objects, expressions, assignment  & \\\hline
  2 & Feb. 3 & function calls, strings, for loops & \\\hline
  3 & Feb. 10 & booleans, function definitions, parameters, lists & \\\hline
  4 & Feb. 17  & while loops, functions continued & \\\hline
  5 & Feb. 24 & grids, tuples & Monday 2/24 Exam 1 \\\hline
  6 & Mar. 2 & turtles & \\\hline
  7 & Mar. 9 & classes and objects & \\\hline
  \multicolumn{4}{|c|}{\textit{Spring Break}}\\\hline
  8 & Mar. 30 & classes and objects (continued) & \\\hline
  9 & Apr. 6 & dictionaries, graphics & Tuesday 4/7 Exam 2\\\hline
  10 & Apr. 13 & graphics, event handling & \\\hline
  11 & Apr. 20 & graphics, files & \\\hline
  12 & Apr. 27 & final project work & \\\hline
%\multicolumn{4}{|c|}{\textit{Thanksgiving Break}}\\\hline
  13 & May 4 & project and presentations & \\\hline
%  14 & May 13 & Project presentations & \\\hline
  14 & May 11 & Final Exam week & \parbox[t]{2in}{See syllabus text.} \\\hline
  
  \hline
\end{longtable}




\end{document}
